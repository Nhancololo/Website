\subsection*{Undergraduate}

\begin{longtable}[l]{@{}p{0.12\textwidth}p{0.88\textwidth}}
2025 & Co-advisor of Armindo Manuel Macamo (\texttt{armindomanuelmacamo2002@gmail.com}), Save University \newline
\newline
\textbf{Title:} Spatio-Temporal Variability of Precipitation in the Limpopo River Basin (1984--2023) \newline
\newline
\textbf{Abstract:} \small The rise in global temperatures driven by climate change is intensifying the hydrological
cycle, leading to a dangerous loop of prolonged droughts and extreme precipitation. This
study focuses on the Limpopo River hydrographic basin, a transboundary basin of
approximately 415,000 $km^2$, shared by South Africa (45\%), Botswana (20\%), Mozambique
(21\%), and Zimbabwe (14\%). The predominantly semi-arid region faces increasing climatic
pressures that threaten the water and food security of its 14 million inhabitants, 90\% of whom
depend on rain-fed agriculture. The degradation of monitoring networks, such as the 68\%
reduction in South African rain gauges since the 1970s, hinders the prediction of extreme
events and adaptive management. In the context of recurrent cyclones (Dineo-2017 and
Eloise-2021) and droughts intensified by the 2023--2024 El Niño, analysing rainfall variability
is crucial for developing effective resilience policies. This research used data from
Mozambique's National Institute of Meteorology, the National Aeronautics and Space
Administration, the Global Runoff Data Centre, and the HydroSHEDS system for the period
from 1984 to 2023. All analyses were conducted in the R/RStudio environment (version
4.3.2) at a 5\% significance level. Precipitation was estimated using ordinary kriging, validated
by leave-one-out cross-validation. The seasonal autoregressive integrated moving average
model was fitted to the time series and validated using (partial) autocorrelation plots, the
Ljung--Box test, and the Akaike information criterion. The precipitation projections for
2024--2033, generated using the SARIMA model, included 95\% confidence intervals. The
results indicated a rainfall regime characterised by high variability, with a well-defined rainy
season from October to April and a heterogeneous spatial distribution, where eastern areas
consistently receive significantly higher precipitation. The most recent period (2019--2023)
recorded the highest volume and greatest variability, suggesting an intensification of extreme
events. Future projections indicate the continuation of this high volatility, without a clear
trend of increasing or decreasing average precipitation. It is concluded that the basin operates
under a cycle of climatic extremes, requiring resilient water and agricultural management
strategies, enhanced early warning systems, and solid transboundary cooperation among the
four sharing countries to ensure food and water security. \newline

\textbf{Keywords:} Climate change; spatio-temporal rainfall variability; Limpopo River hydrographic basin; water security; SARIMA. \\
\end{longtable}